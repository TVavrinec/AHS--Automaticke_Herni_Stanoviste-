% Vysázení stránky s rozšířeným abstraktem
% (týká se pouze bc. a dp. prací psaných v angličtině, viz Směrnice rektora 72/2017)
\cleardoublepage
\noindent
{\large\sffamily\bfseries\MakeUppercase{Rozšířený abstrakt}}
\\
Výtah ze směrnice rektora 72/2017:\\
\emph{Bakalářská a diplomová práce předložená v angličtině musí obsahovat rozšířený abstrakt v češtině
nebo slovenštině (čl. 15). To se netýká studentů, kteří studují studijní program akreditovaný v
angličtině.}
(čl. 3, par. 7)\\
\emph{Nebude-li vnitřní normou stanoveno jinak, doporučuje se rozšířený abstrakt o rozsahu přibližně 3
normostrany, který bude obsahovat úvod, popis řešení a shrnutí a~zhodnocení výsledků.}
(čl. 15, par. 5)