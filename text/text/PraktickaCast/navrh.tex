Celé zařízení je rozděleno na základní jednotku a případné moduly které zajistí nové herní možnosti.
Například může jít o připojení úložného prostoru nebo zvukového modulu, který poskytne jak plnohodnotný zvukoví výstup tak vstup.

\subsection{uživatelské požadavky}
Jde o statické zařízení sloužící jako herní stanoviště.
Vyžaduje tedy mobilitu ale jen v rámci transportu na místo hry a spt, nikoliv v rámci samotné hry.

Zařízení bude mít dva světelné kruhy složené z inteligentních RGB LED.
Jeden radiální a druhý axiální na horní straně zařízení sloužící primárně jako odezva pro hráče na malou vzdálenost, např. při zadávání hesla.
Radiální kruh, také v horní části zařízení slouží naopak pro signalizace na delší vzdálenost, takový maják.

Uvnitř axiálního světelného kruhu se bude nacházet tzv. tlaková plocha.
Jedná se o ovládací prvek podobný dotykové ploše, s tím rozdílem že je schopen měřit i silu která na něj působí.

Aby bylo stanoviště reálně použitelné při hře musí celou hru vydržet na baterii.
Není ojedinělé aby měla outdoorová hra čtyři pět hodin bez přestávky.
Plus je nutná nějaká rezerva a čas na nastavování.
Pochopitelně je čas který zařízení zvládne běžet z baterie, silně závisí na činnosti, ale nebylo mi zrovna ideální, kdyby byla baterka, nějak výrazně omezující.
Výdrž na jedno nabití bych tedy chtěl směřovat alespoň na pět hodin.

Vzhledem k plánu připojovat moduly je nutné vyřešit jak se to budě dělat.
Bylo by ideální kdyby si mohl uživatel říct co bude hrát za hru a podle toho si sám připojil moduly které potřebuje.
Tomuto určitě nechceme bránit ale přímo to podporovat nese řadu problému jak ze strany konektoru a mechaniky, tak ze strany softwaru.
Konektor by totiž musel být ideálně beznástrojové rozpojitelný a opětovně spojitelný a přitom dostatečně pevný aby se zařízení mechanicky chovalo jako jeden celek.
Takový konektor je ale poměrně složité udělat tak aby byl spolehliví a tak jde v tuto chvíli spíš o hudbu budoucnosti.
Ze softwarového pohledu jde pak o problém z duvodu detekce konkrétních modulu a hlavně o otázku jak se chovat k modulum které jsou potenciálně záměnné.
Dejme tomu že máme modul klávesnici a modul dvířka.
Dvířka jsou původně primárně úložný prostor, díky detekci zavření je lze ale použít i jako velmi pohodlná tlačítka a v některých hrách se proto používají jen jako tlačítka.
Potenciální modul klávesnice je ovšem jen suma tlačítek.
Při vytváření konkrétní hry na míru modulum které herní návrhář má zrovna k dispozici je tento problém nepodstatný protože sám návrhář rozhodně co má jak být.
Ale ve chvíli kdy je ale o hru navrženou pro jinou kombinaci modulu nastává problém jak rozhodnout zda se dají dvířka použít místo klávesnice nebo ne.
Abychom se všem temto problémům, alespoň prozatím vyhnuli, rozhodli jsme že doplnění, či výměna modulu, pujde jen při servisním zásahu.

Některé hry vyžadují tak velké herní území že by na komunikaci mezi stanovišťmi už nestačila WiFi ani Bluetooth, které AHS jinak využívá ke komunikaci.  
Proto bude mít AHS možnost připojení k mobilní síti a tedy připojení k internetu.
Tím se za prvé rozšíří dosah AHS, všude kde je mobilní pokrytí a také přibude další metoda jak se stanovištěm komunikovat.
V rámci tohoto komunikačního modulu bude možné používat navíc i GNSS \footnote{Global Navigation Satellite System}.
Vzhledem k faktu že se přece jen nejedná o systém, který by využila většina her a zároveň je poměrně drahý došli jsme k rozhodnutí, mít jej jen jako doplnitelný modul.
Protože se jedná o modul který zprostředkovává komunikaci se světem je pravděpodobné že bude potřebovat převádět výrazně větší množství dat než běžný modul.
Primárně z tohoto důvodu je tento modul připojen na samostatném konektoru.
Potřebné antény budou už v základním zařízení ale samotný modul spadá do doplňkové výbavy.

V řadě případu je užitečné mít možnost zvuková zpětné vazby.
Ideální by bylo moci přehrávat libovolnou nahrávku, většinou ale stačí jednoduchý ton jako řekněme potvrzení na zadané heslo.
Možnost přehrávat plnohodnotnou nahrávku proto odsouváme jako možný doplňkový modul a v základní jednotce pro jednoduchost postačí piezoměnič. 

% \begin{itemize}
%     \item Dva LED kruhy jeden radiální na horní straně zařízení a jeden axiální v horní části zařízení
%     \item Tlaková plocha
%     \item Výdrž na baterii alespoň pět hodin aby bylo možné organizovat čtyřhodinovou hru s rezervou na nastavování atd.
%     \item Servisně snadné připojený komunikační/GPS modul uvnitř zařízení
%     \item Servisně snadné připojená dvířka a jiné moduly vně základního zařízení
%     \item Základní zvuková odezva
% \end{itemize}

\subsection{Struktura elektroniky základní jednotky}

Elektronika je rozdělena na tři samostatné PCB.
Jde o HlavníDeska na které je většina elektroniky, o~desku s~hlavním uživatelským rozhraním (LedDeska) a~o~obslužnou desku s~minimalistickým uživatelským rozhraním pro neherní obsluhu (MiniUI).

\subsubsection{LedDesku}
Jak název napovídá na LedDesce se nacházejí oba světelné kruhy, mimo to je zde i elektronika pro snímání tlakové plochy, tedy LDC1x14 \cite{LDC1614} a jeho snímací cívky.
Právě snímaní tlakové plochy je jeden z podstatných duvodu oddělení této desky od zbytku, zabere totiž docela dost prostoru.

\begin{itemize}
    \item Axiální LED kruh z 60ti RGB LED WS2812
    \item Zadní radiální LED kruh z 60ti RGB LED WS2812
    \item LDC1614 nebo LDC1314 se čtyřmi snímacími cívkami pro snímání tlakové plochy
    \item konektor na propojení s HlavníDeskou
\end{itemize}

\subsubsection{MiniUI}
Kvuli aktuální představě mechanické konstrukce není úplně dobře možné mít toto minimalistické uživatelské rozhraní na HlavníDeska.
Proto sme se rozhodli jej vyseparovat na samostatnou destičku, na které bude jen pár tlačítek a dvě signalizační LED.

\begin{itemize}
    \item RESET tlačítka
    \item BOOT tlačítka
    \item zapínací tlačítko
    \item dvě uživatelská tlačítka 
    \item dvě uživatelské ledky
\end{itemize}

\subsubsection{HlavníDeska}
Na HlavníDesce je většina systému základního zařízení.

Protože máme dlouholetým zkušenostem s mikrokontrolery ESP32, rozhodovali jsme jen mezi konkrétními čipy z této rodiny.
% Dnes už tedy jen většina, mikrokontroleru této rodiny má integrovanou WiFi a Bluetooth periferii, kterou AHS určitě využije.
Aktuální plán na tvorbu vysoko-úrovňového návrhu konkrétních her je používat skriptovací jazyk JavaScript, považujeme za nevhodné provádět tento návrh na úrovní jazyka C/C++.
Abychom tohoto mohli dosáhnout, musíme provozovat nějaký interpreter JavaScriptu a z aktuálně dostupných možností, považujeme za nejideálnější Jaculus \cite{Jaculus}.
Jaculus dělá řadu věcí paralelně a je tím pádem pro jeho chod ideální mít více jader procesoru, což aktuálně znamená tři možnosti ESP32, ESP32S2 a ESP32S3.
ESP32S2 nemá Bluetooth, který bychom rádi měli k dispozici.
ESP32S3 je téměř jen vylepšená verze staršího ESP32, oproti tomu mu sice chybí pár možností, ty ale v AHS nepotřebujeme využívat a naopak se nám hodí vyšší výpočetní výkon, rychlejší paměť a více piny.
AHS tedy ovládá mikrokontroler ESP32S3.

Zdroj AHS je tvořen dvěma LiIon články 18650 v paralelním uspořádání.
Paralelní uspořádání jsme zvolili aby nebylo nutné řešit balancování článků.

Aby nebylo možné softwarově baterii podvybít má AHS systém, který celé zařízení vypne v případě kdy dojde k vybití baterie pod 2.8V.
Pochopitelně software by měl vybitou baterii zaznamenat mnohem dřív a chovat se podle toho, např. neumožnit spustit hru s baterii na napětím 3.0V.

Baterii je pochopitelně nutno nabijet a bylo by velmi nepohodlné kvuli tomu muset vytahovat články ze zařízení.
Proto je na HlavníDesce i nabíjecí elektronika.
Navíc aby se minimalizoval čas nabíjení zařízení podporuje standard PowerDelivery a to až do napětí 21V.

Protože různé periferie vyžadují různá napájecí a komunikační napětí jsou na HlavníDeska hned čtyři napájecí větvě.
\begin{itemize}
    \item VCC, napětí baterie sloužící jako zdroj pro ostatní napájecí větvě a pro napájení komunikačního modulu. 
    \item Napětí \(3.3\-[V]\) na napájení logické části celého základního zařízení.
    \item Napětí \(5.0\-[V]\) pro LedDesku a externí moduly
    \item Napětí \(1.8\-[V]\) pro napájení napěťových převodníků sloužících na komunikaci s komunikačním modulem 
\end{itemize}
Napětí jednotlivých větví to u \(3.3\-[V]\) a \(1.8\-[V]\) tvořeno pomocí LDO.
Na vytvoření pěti voltové větvě je ale potřeba spínaný zdroj a to primárně ze dvou důvodů.
Za prvé protože napětí baterie ze které se tato větev napájí má nižší napětí a je jej tedy třeba vyspínat na napětí vyšší.
Za druhé tento zdroj poskytuje do systému mnohem větší proudy než druhé dvě větve a bylo by tedy vhodné, ho použít i v případě použití sériového řazení článků.

Na hlavníDesce je také řada konektorů sloužící pro připojení ostatních systémů.
Jde o konektory na:
\begin{itemize}
    \item propojení s LedDeskou                                             % samostatný objekt
    \item připojení MiniUI                                                  % samostatný objekt
    \item komunikační modul (M2 Konektor umožňuje použít různé moduly)      % externě definovaný objekt
    \item externí moduly                                                    % samostatný objekt
    \item USB-C (nabíjení a programování AHS)                               % externě definovaný objekt
    \item programátor                                                       % samostatný objekt
\end{itemize}
Do konektorů by se asi dal zařadit i držák na dva LiIon články 18650.

Za zmínku také stojí přítomnost piezoměniče pro jednoduchou zvukovou odezvu. 

\subsubsection{Propojení HlavníDesky a LedDesky}
Mezi HlavníDeskou a LedDeskou je třeba převést napájení a několik signálů.
LedDeska vyžaduje na konektoru přítomnost dvou napájecích větví \(5\-[V]\) pro světelné kruhy a \(3.3\-[v]\) pro snímání tlakové plochy.
Protože do LEDek může téct proud až 5A a být zároveň i docela rychle spínaný, považujeme za rozumné oddělit napájecím větvím zem.
Oddělení je tedy provedeno už na konektoru HlavníDesky.

Na samotné propojení jsme se rozhodli použít FFC kabel s roztečí \(0.5\-[mm]\).
Jednim vodičem takovéhoto kabelu zle vést proud maximálně \(0.4\-[A]\) \cite{FFC-konektor}.
Protože ale potřebujeme dodat proud až 5A použijeme 13 vodičů vedle sebe jakožto nejmenší počet který přenese požadovaný proud v rámci daných mezí.

Mimo napájení je tímto propojením veden i signál s daty pro světelné kruhy a I2C sběrnice s interruptem pro připojení čipu LDC1614 \cite{LDC1614}.

Vzhledem k počtu potřebných vodičů (konkrétně 32) jsme se rozhodli použít velmi běžný FFC konektor se 40 kontakty, s tím že zbylé kontakty se mohou hodit v budoucnu.

\subsubsection{Modulový konektor}
Modulovým konektorem je vedeno \(5\-[V]\) jako napájení pro moduly a UART s interruptem pro komunikaci.
Nad volbou komunikační sběrnice jsme strávili poměrně dost času přemýšlením.
Původně jsme uvažovali o využití RS485 jakožto odolné sběrnice u které by v případ potřeby nemusel být problém ani delší kabel.
RS485 má ale nevýhodu v tom že potřebuje dodatečný hardware, kterému bychom se hlavně na modulech rádi vyhnuli.
Stejný problém nastel u CANu i USB které by navíc mělo výhodu kompatibility se velkým množstvím hotových zařízení.

V první uvaze o UARTu jsme jej nejprve zavrhli kvuli potenciální náročnosti na přeposílání dat mezi moduly.
Při standardním použití, bychom totiž moduly řadili za sebe.
Prvnímu modulu by tak chodili data pro všechny ostatní moduly a musel by je přeposílat dál, což by stálo nezanedbatelné množství procesorového času.
V jisté chvíli jsme ale narazili na nestandardní komunikaci pomocí UARTu implementované v projektu Servio \cite{Servio}.
Tato implementace používá UART jako sběrnici.
Namísto standardního použití pro komunikace jeden s jedním tak muže komunikovat jeden s více.
Na tomto řešení je výhodné že nevyžaduje žádný dodatečný hardware a prakticky každý dnešní mikrokontroler je možné k této sběrnici velmi snadno připojit.
Ve srovnání s RS485 je sice mnohem méně odolná proti rušení, ale uvnitř zařízení nebude linka vedena na víc jak malé desítky centimetru.
Komunikace na delším kabelu je pak jednoduše nahraditelné bezdrátovou komunikací a není tak potřebné aby to nativně umožňovala tato sbernice.
Každopádně v případě potřeby delšího kabelu je možné navrhnout externí modul, který s této sbernic velmi jednoduše udělá RS485.
Alternativně by se pro komunikaci na delším kabelu dalo použít USB které je společně s nabíjením přivedeno na USB-C.

Všechny moduly jsou tedy připojeny na jeden RX pin AHS.
Proto musí firmware AHS zajistit aby dva moduly nevysílali současně.
Aby se zabránilo možným zkratum, jako ochranu má každý modul své piny UARTu připojeny přes rezistor \(180\-[\Omega]\).
Interrupt pin modulů se naopak chová jako open-collector a na straně hlavního zařízení je na něj tedy připojen pull-up rezistor.
Abychom alespoň trochu zvýšili odolnost linky proti rušení přidáme na přijímací stranu pull-up rezistor.
Cílem je zvýšení komunikačního proudu aby se případný proud vyvolaný rušením dal jednodušeji zanedbat.
V neposlední řadě jsou mají všechny piny na konektoru ESD ochranu.

\subsubsection{Konektor programátor}
Zařízení se dá jednoduše programovat přes USB-C, tento kanál se ale dá softwarově narušit a pro takové případy je tu konektor na programátor.
Jde o šest plošek na které se programátor připojuje pomocí pogo-pin.
Programátor sice obsahuje jen jednoduchou elektroniku která by mohla být i přímo v elektronice AHS ale ve většině případu by byla zbytečná.
Ve chvíli kdy by byla potřeba tak je stejně nutná odborná obsluha a pro tu není problém použít programátor.

\subsubsection{USB-C}
Jako napájecí a programovací konektor je použito USB-C.
Díky němu je možné podporovat standard PowerDelivery využitý pro zrychlení nabíjení.
Konektor je ale použit i na pohodlnější programování zařízení, pez potřeby programátoru.

\subsubsection{Konektor komunikačního modulu}
Pro připojení komunikačního modulu jsme zvolili konektor M2 typ-B jakožto standard pro tyto moduly.
Díky tomuto konektoru můžeme jednoduše připojit různé LTE a GNSS moduly.


% Na konektor, je z ESP vyvedeno: 
% -   I2C
% -   celý UART
% -   FUL_CARD_POWER_OFF
% -   RESET
% -   WoWWAN (interrupt)

% a mimo ESP je na konektoru:
% -   napájení přímo z baterie
% -   SIM

% Pin-out M2 konektoru z https://files.waveshare.com/upload/0/02/SIM7600X-M2_Hardware_Design_V1.01.pdf
% ![](img/SIM7600G-H-M.2-PIN.png)


% ## mechanická část
% Základ AHS je mechanicky krátký válec s tlakovou plochou na horní ploše.
% Kolem tlakové plochu je umístěn axiální světelný kruh a v horní části válcové plochy je radiální světelný kruh.

% - Tělo minimálně v prototypu tisknuté
% - Vyzkoušet různé provedení kontaktní plochy
%     - Stejně jako na předchozí verzi FR4 ale tlustší aby se tolik nekroutila
%     - PCB s hliníkovým jádrem + zalití do epoxidu nebo jiné dostatečně průsvitné hmoty
% - Konektor externích modulu, USB-C a mini UI překrýt krytkou (ochrana před bordelem a náhodným mačkáním na tlačítka) 

% ![](img/dvirka.png)
