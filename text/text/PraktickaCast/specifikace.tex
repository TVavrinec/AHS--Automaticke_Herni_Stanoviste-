Celé zařízení je rozděleno na základní jednotku a případné moduly které zajistí nové herní možnosti.
Například může jít o připojení úložného prostoru nebo zvukového modulu, který poskytne jak plnohodnotný zvukoví výstup tak vstup.

\subsection{uživatelské požadavky}
Jde o statické zařízení sloužící jako herní stanoviště.
Vyžaduje tedy mobilitu ale jen v rámci transportu na místo hry a spt, nikoliv v rámci samotné hry.

Zařízení bude mít dva světelné kruhy složené z inteligentních RGB LED.
Jeden radiální a druhý axiální na horní straně zařízení sloužící primárně jako odezva pro hráče na malou vzdálenost, např. při zadávání hesla.
Radiální kruh, také v horní části zařízení slouží naopak pro signalizace na delší vzdálenost, takový maják.

Uvnitř axiálního světelného kruhu se bude nacházet tzv. tlaková plocha.
Jedná se o ovládací prvek podobný dotykové ploše, s tím rozdílem že je schopen měřit i silu která na něj působí.

Aby bylo stanoviště reálně použitelné při hře musí celou hru vydržet na baterii.
Není ojedinělé aby měla outdoorová hra čtyři pět hodin bez přestávky.
Plus je nutná nějaká rezerva a čas na nastavování.
Pochopitelně je čas který zařízení zvládne běžet z baterie, silně závisí na činnosti, ale nebylo mi zrovna ideální, kdyby byla baterka, nějak výrazně omezující.
Výdrž na jedno nabití bych tedy chtěl směřovat alespoň na pět hodin.

Vzhledem k plánu připojovat moduly je nutné vyřešit jak se to budě dělat.
Bylo by ideální kdyby si mohl uživatel říct co bude hrát za hru a podle toho si sám připojil moduly které potřebuje.
Tomuto určitě nechceme bránit ale přímo to podporovat nese řadu problému jak ze strany konektoru a mechaniky, tak ze strany softwaru.
Konektor by totiž musel být ideálně beznástrojové rozpojitelný a opětovně spojitelný a přitom dostatečně pevný aby se zařízení mechanicky chovalo jako jeden celek.
Takový konektor je ale poměrně složité udělat tak aby byl spolehliví a tak jde v tuto chvíli spíš o hudbu budoucnosti.
Ze softwarového pohledu jde pak o problém z duvodu detekce konkrétních modulu a hlavně o otázku jak se chovat k modulum které jsou potenciálně záměnné.
Dejme tomu že máme modul klávesnici a modul dvířka.
Dvířka jsou původně primárně úložný prostor, díky detekci zavření je lze ale použít i jako velmi pohodlná tlačítka a v některých hrách se proto používají jen jako tlačítka.
Potenciální modul klávesnice je ovšem jen suma tlačítek.
Při vytváření konkrétní hry na míru modulum které herní návrhář má zrovna k dispozici je tento problém nepodstatný protože sám návrhář rozhodně co má jak být.
Ale ve chvíli kdy je ale o hru navrženou pro jinou kombinaci modulu nastává problém jak rozhodnout zda se dají dvířka použít místo klávesnice nebo ne.
Abychom se všem temto problémům, alespoň prozatím vyhnuli, rozhodli jsme že doplnění, či výměna modulu, pujde jen při servisním zásahu.

Některé hry vyžadují tak velké herní území že by na komunikaci mezi stanovišťmi už nestačila WiFi ani Bluetooth, které AHS jinak využívá ke komunikaci.  
Proto bude mít AHS možnost připojení k mobilní síti a tedy připojení k internetu.
Tím se za prvé rozšíří dosah AHS, všude kde je mobilní pokrytí a také přibude další metoda jak se stanovištěm komunikovat.
V rámci tohoto komunikačního modulu bude možné používat navíc i GNSS \footnote{Global Navigation Satellite System}.
Vzhledem k faktu že se přece jen nejedná o systém, který by využila většina her a zároveň je poměrně drahý došli jsme k rozhodnutí, mít jej jen jako doplnitelný modul.
Protože se jedná o modul který zprostředkovává komunikaci se světem je pravděpodobné že bude potřebovat převádět výrazně větší množství dat než běžný modul.
Primárně z tohoto důvodu je tento modul připojen na samostatném konektoru.
Potřebné antény budou už v základním zařízení ale samotný modul spadá do doplňkové výbavy.

V řadě případu je užitečné mít možnost zvuková zpětné vazby.
Ideální by bylo moci přehrávat libovolnou nahrávku, většinou ale stačí jednoduchý ton jako řekněme potvrzení na zadané heslo.
Možnost přehrávat plnohodnotnou nahrávku proto odsouváme jako možný doplňkový modul a v základní jednotce pro jednoduchost postačí piezoměnič. 


% \begin{itemize}
%     \item Dva LED kruhy jeden radiální na horní straně zařízení a jeden axiální v horní části zařízení
%     \item Tlaková plocha
%     \item Výdrž na baterii alespoň pět hodin aby bylo možné organizovat čtyřhodinovou hru s rezervou na nastavování atd.
%     \item Servisně snadné připojený komunikační/GPS modul uvnitř zařízení
%     \item Servisně snadné připojená dvířka a jiné moduly vně základního zařízení
%     \item Základní zvuková odezva
% \end{itemize}

\subsection{Struktura elektroniky základní jednotky}

Elektronika je rozdělena na tři samostatné PCB.
Jde o HlavníDeska na které je většina elektroniky, o~desku s~hlavním uživatelským rozhraním (LedDeska) a~o~obslužnou desku s~minimalistickým uživatelským rozhraním pro neherní obsluhu (MiniUI).

\subsubsection{LedDesku}
Jak název napovídá na LedDesce se nacházejí oba světelné kruhy, mimo to je zde i elektronika pro snímání tlakové plochy, tedy LDC1x14 \cite{LDC1614} a jeho snímací cívky.
Právě snímaní tlakové plochy je jeden z podstatných duvodu oddělení této desky od zbytku, zabere totiž docela dost prostoru.

\begin{itemize}
    \item Axiální LED kruh z 60ti RGB LED WS2812
    \item Zadní radiální LED kruh z 60ti RGB LED WS2812
    \item LDC1614 nebo LDC1314 se čtyřmi snímacími cívkami pro snímání tlakové plochy
    \item konektor na propojení s HlavníDeskou
\end{itemize}

\subsubsection{Mini UI}
Kvuli aktuální představě mechanické konstrukce není úplně dobře možné mít toto minimalistické uživatelské rozhraní na HlavníDeska.
Proto sme se rozhodli jej vyseparovat na samostatnou destičku, na které bude jen pár tlačítek a dvě signalizační LED.

\begin{itemize}
    \item RESET tlačítka
    \item BOOT tlačítka
    \item zapínací tlačítko
    \item dvě uživatelská tlačítka 
    \item dvě uživatelské ledky
\end{itemize}

\subsubsection{HlavníDeska}
Na HlavníDesce je většina systému základního zařízení.
Aby bylo možné určit chování celého zařízení je potřebný nějaký mikrokontroler.
Protože máme dlouholetým zkušenostem s mikrokontrolery ESP32, rozhodovali jsme jen mezi konkrétními verzemi z této rodiny.
% Dnes už tedy jen většina, mikrokontroleru této rodiny má integrovanou WiFi a Bluetooth periferii, kterou AHS určitě využije.
Aktuální plán na tvorbu vysoko-úrovňového návrhu konkrétních her je používat skriptovací jazyk JavaScript, považujeme za nevhodné provádět tento návrh na úrovní jazyka C/C++.
Abychom tohoto mohli dosáhnout, musíme provozovat nějaký interpreter JavaScriptu a z aktuálně dostupných možností, považujeme za nejideálnější Jaculus \cite{Jaculus}.
Jaculus dělá řadu věcí paralelně a je tím pádem pro jeho chod ideální mít více jader procesoru, což aktuálně znamená tři možnosti ESP32, ESP32S2 a ESP32S3.
ESP32S2 nemá Bluetooth, který bychom rádi měli k dispozici.
ESP32S3 je téměř jen vylepšená verze staršího ESP32, oproti tomu mu sice chybí pár možností, ty ale v AHS nepotřebujeme využívat a naopak se nám hodí vyšší výpočetní výkon, rychlejší paměť a více piny.
AHS tedy ovládá mikrokontroler ESP32S.




\begin{itemize}
    \item micro kontroler
    \item systém na dohodnutí PD
    \item bzučák (bez oscilátorové)
    \item Power manager
    \begin{itemize}
        \item připojení na dva paralelní LiIon články 18650
        \item hardwarová ochrana podvybití
        \item hardwarové řešení zapínání s ovládáním vyvedeným na MiniUI
        \item nabíječka
        \item step-up na 5V
        \item LDO na 3V3
        \item LDO na 1V8
    \end{itemize}
    \item Konektory
    \begin{itemize}
        \item LedDeska
        \item mini UI
        \item komunikační modul
        \item externí moduly
        \item USB-C (nabíjení a primární programování)
        \item případné zbylé piny
    \end{itemize}
\end{itemize}

\subsubsection{Popis jednotlivých bodů}
\paragraph{Micro kontroler}
všemu vládne je ESP32S3 \cite{ESP32S3-WR} až na to že něčemu ne.

\paragraph{Dohodnutí PD}
Komunikaci PD zajišťuje čip AP33772 \cite{AP33772} který je přes \(I2C\) připojený na ESP.

\paragraph{Bzučák}
Piezzo připojené mezi dva piny ESP aby bylo možné programově nastavit frekvenci a částečně i hlasitost.

\paragraph{Power manager}
\subparagraph{Zapínání}
Umožňuje uživateli zařízení zapínání tlačítkem.
Vypnutí je možné provést softwarově z ESP.

\subparagraph{Step-up na 5V}
Čip TPS61088 \cite{TPS61088} zajišťuje napájení světelným kruhům a externím modulům.
Je tedy schopen dodat proud pro oba světelné kruhy a moduloví konektor což znamená cca \(7\-[A]\).

\subparagraph{LDO na 3V3}
Zajišťuje napájení pro ESP32S3, LDC1x14 a PD sink AP33772 \cite{AP33772}.
Zároveň je z tohoto napětí odvozena větev 1.8V.
Proudově je tedy sto dodat cca \(1\-[A]\).

\subparagraph{LDO na 1V8}


\subparagraph{Nabíječka}
Nabíjení probíhá přes USB-C, nabíječka je proto schopna provozu z 5V.
Aby se ale zařízení dalo nabít rychleji je možno použít standard PowerDelivery s napětí až 21V.
Na nabíjení je proto použit chip BQ24179YBGR \cite{BQ24179}.

\subparagraph{Hw ochrana podvybití}
Kvuli ochraně baterie se zařízení vypne vypne při jejim vybití pod 2.8V.
Podle \cite{PanasonicLiOntReport} napětí na článek nesmí klesnout pod 2.3V. 
AHS k tomu volbou chipu BQ298012 \cite{BQ298012}, přidává rezervu 0.45V 



% ## Konektory
% ### USB-C
% D+ D- jsou napojeny přímo na ESP a CC piny jsou napojeny na AP33772(to přes I2C na ESP).

% ### Konektor komunikačního modulu
% Na konektor, je z ESP vyvedeno: 
% -   I2C
% -   celý UART
% -   FUL_CARD_POWER_OFF
% -   RESET
% -   WoWWAN (interrupt)

% a mimo ESP je na konektoru:
% -   napájení přímo z baterie
% -   SIM

% Pin-out M2 konektoru z https://files.waveshare.com/upload/0/02/SIM7600X-M2_Hardware_Design_V1.01.pdf
% ![](img/SIM7600G-H-M.2-PIN.png)

% ### Konektor na LedDesku

% <-1--13-> 5V

% <----14-> LED-DO

% <-15-27-> GND

% <----28-> 3V3

% <----29-> LDC-GND

% -----30-> LDC-INT

% <----31-> SDA

% <----32-- SCL


% Pull-upy na I2C a na LDC-INT mají 4.7kΩ.
% LDC-GND je na hlavní desce normální GND ale kvůli velkým proudum do ledek je v kabelu a na Led desce vedena odděleně.
% Konektorem na 5V proteče alespoň 5A.

% ### Moduloví konektor

% ---> USART-RX

% <--> 5V

% <--> GND

% <--- USART-TX

% <--> GND

% <--> 5V

% ---> INI

% Všechny moduly jsou připojeny na jeden RX pin AHS.
% Proto musí firmware AHS zajistit aby dva moduli nevysílali současně!
% Každý modul má TX připojen přes rezistor 180Ω jako ochranu.

% Interrupt na modulech se chová jako open-collector což je zajistit hardwarově.

% AHS zajišťuje odolnost proti rušení na všech vstupních vodičích pomocí 4.7kΩ pull-up a odolnost proti zkratu na datových výstupních vodičích pomocí 180Ω.
% Zároveň zajistí ESD ochranu všem vodičům co vedou z/do AHS.

% Konektor je zároveň sto dodat napájení 5V s proudem v součty 2A.

% Komunikace muže být vedena kabelem o délce až 25cm při komunikační rychlosti 5Mbps.
% *Při dalším prodlužování linky bude třeba vodiče RX a TX vlnově přizpůsobit a doplnit linku terminátory*



% ## mechanická část
% Základ AHS je mechanicky krátký válec s tlakovou plochou na horní ploše.
% Kolem tlakové plochu je umístěn axiální světelný kruh a v horní části válcové plochy je radiální světelný kruh.

% - Tělo minimálně v prototypu tisknuté
% - Vyzkoušet různé provedení kontaktní plochy
%     - Stejně jako na předchozí verzi FR4 ale tlustší aby se tolik nekroutila
%     - PCB s hliníkovým jádrem + zalití do epoxidu nebo jiné dostatečně průsvitné hmoty
% - Konektor externích modulu, USB-C a mini UI překrýt krytkou (ochrana před bordelem a náhodným mačkáním na tlačítka) 

% ![](img/dvirka.png)
